%%%%%%%%%%%%%%%%%%%%%%%%%%%%%%%%%%%%%%%%%
% fphw Assignment
% LaTeX Template
% Version 1.0 (27/04/2019)
%
% This template originates from:
% https://www.LaTeXTemplates.com
%
% Authors:
% Class by Felipe Portales-Oliva (f.portales.oliva@gmail.com) with template 
% content and modifications by Vel (vel@LaTeXTemplates.com)
%
% Template (this file) License:
% CC BY-NC-SA 3.0 (http://creativecommons.org/licenses/by-nc-sa/3.0/)
%
%%%%%%%%%%%%%%%%%%%%%%%%%%%%%%%%%%%%%%%%%

%----------------------------------------------------------------------------------------
%	PACKAGES AND OTHER DOCUMENT CONFIGURATIONS
%----------------------------------------------------------------------------------------

\documentclass[
	11pt, % Default font size, values between 10pt-12pt are allowed
]{fphw}

% Template-specific packages
\usepackage[utf8]{inputenc} % Required for inputting international characters
\usepackage[T1]{fontenc} % Output font encoding for international characters
\usepackage{mathpazo} % Use the Palatino font

\usepackage{graphicx} % Required for including images

\usepackage{booktabs} % Required for better horizontal rules in tables

\usepackage{listings} % Required for insertion of code

\usepackage{enumerate} % To modify the enumerate environment

\usepackage{color}
\usepackage{todonotes}
\usepackage{scrextend}
\usepackage{caption}
\usepackage{nameref}
\usepackage{hyperref}
\usepackage{minted}
\usepackage{framed}
\usepackage{multirow}
\usepackage{pifont}
\usepackage{float}
\usepackage{amsmath}

\restylefloat{table}

\definecolor{color_background}{rgb}{0.98,0.98,0.98}
\definecolor{shadecolor}{rgb}{0.98,0.98,0.98}

\newenvironment{code}
    {\captionsetup{
        type=listing,
        skip=2pt,
        belowskip=15pt
        }}
    {}

\setminted{
    linenos=true, 
    frame=lines,
    breaklines=true,
    bgcolor=color_background,
    }
\usemintedstyle{one-dark}
\setmintedinline{breaklines}

%----------------------------------------------------------------------------------------
%	ASSIGNMENT INFORMATION
%----------------------------------------------------------------------------------------

\title{Project: Non-lexical sounds in dialogue utterances} % Assignment title

\date{November 30th, 2022} % Due date

\author{Dominik Künkele}

\institute{University of Gothenburg} % Institute or school name

\class{Machine Learning 2 (LT2326)} % Course or class name

%----------------------------------------------------------------------------------------

\begin{document}

\maketitle % Output the assignment title, created automatically using the information in the custom commands above

%----------------------------------------------------------------------------------------
%	ASSIGNMENT CONTENT
%----------------------------------------------------------------------------------------

\section*{Background}
In one of their papers, Edlund and his colleagues describe human interaction with machines using metaphors on a spectrum. On the one side of the spectrum is the so called \emph{interface metaphor}. The interface of the machine is built in a way that users are aware, they are talking to a machine. Consequently, they also adapt their language, and use more command like utterances (e.g. "Call John", "Set the timer") as they would fill slots in an imaginary web form. On the other end of the spectrum resides the \emph{human methapor}. Here, the users don't really know, they are talking to a machine and therfore use a "normal" language in the sense of utterances, they would uses in a human-human dialogue. While the interface metaphor is still used very commonly, many dialogue systems like \emph{Alexa} or \emph{Siri} lie somewhere in between these extremes. The human metaphor is implemented rarely and is seen more often in science fiction.

Nevertheless, there are a few strong reasons for using the human metaphor and let the users talk in natural language. Natural language is
\begin{description}
    \item[easy to use.] Since we use natural language all the time in all human interactions, it is very natural for us, to also use it in machine interaction.
    \item[flexible.] Natural language allows us to express everything we want to express. We can express for instance thoughts, feelings or facts with different certainities or also for example talk about things that were, things that are, and some things that have not yet come to pass. There are only very few things in a human mind that cannot be represented in natural language (partly in combination with mimic and gestures).
    \item[resilient to error handling.] Furthermore, it allows us, to correct things we said or also specify certain parts more if we realize the listener is not understanding very easily. In the role of the listener, we can also verify, what we understood. Both can be done in various ways, by for instance rephrasing the utterance, repeating all or parts of it or even ask questions about it.
    \item[enjoyable] Using Natural Language is finally also much more enjoyable than using a command like language. This of course depends on the person, but in general, humans evolved to use and interact in this language.
\end{description}

The goal is now, not to create a machine that is very close to humans, but to create "[a] machine that acts human enough that we respond to it as we respond to another human" as Cassel puts it in his paper in 2007.

There are endless factors that could define, \emph{what} human-like interaction is. They inlcude for example a change in pitch (upwards, to signal turn-keeping; downwards, to signla turn-yielding) or choices of words ("This is the automated booking system" will obviously signal the user that it speaks to a machine, while "Hello" does not). One big factor are non-lexical sounds (NLS) like hesitations ("uh", "hmm", ...) or repetitions of words/n-grams ("if I'm home then I I definitely watch her"). This could be famously observed in the \emph{Google Duplex} demo in 2018, when a machine was calling a hair saloon using NLS.

This project aims to make machine utterances more human-like, by adding NLS to an utterance at natural positions. Additionally, the utterances could also be enhances, by adding repetitions of n-grams.

\section*{Data resource}
As a resource for this project, I use the \emph{Switchboard corpus}. There are various dialogue corpora published, but the Switchboard corpus contains NLS in a mostly structured way, while keeping the rest of the dialogue mostly undistilled, but annotated and as close as possible to the recordings. This means that the corpus inludes
\begin{description}
    \item[non-grammatical sentences:] "yeah i think it is too it's gonna get better"
    \item[repetitions of n-grams:] "oh she didn't she didn't do something"
    \item[annotated partial words:] "bec- because we're so"
    \item[anomalies of words with correct word] "bettle/better"
\end{description}

Furthermore, the corpus classifies the NLS based on there sound into thirteen classes: ah, eh, hm, huh, huh-uh, hum-um, ooh, uh, uh-huh, uh-hum, uh-oh, um, um-hum. These classes will then be used for predicting the NLS in the generated sentences.

I cleansed the corpus, by removing annotations, \emph{silences} and laughters, replacing partial with complete words and NLS with a specific token per class. In the end, I could extract 247 123 utterances. The length of these utterances ranges from 1 to 81 words. Since I only used each utterance by itself to generate NLS and was not using any realtions in the dialoue, I also excluded utterances shorter than than words. That yielded me around 150 000 usable utterances. Around 70 000 of these utterances contain at least one NLS.

Almost half of the utterances contain a repetition. Most of the repetitions are only on word, but I could identify also repetitions of up to 9-grams.

\section*{Methods}
The idea of this project was to predict if and which NLS should follow each word in a sentence. The model consists of (A) an embedding layer for tokens and POS, (B) encoder layers for tokens and POS and (C) a classifier that classifies the NLS for the output of the encoding at each encoding step.

\subsection*{(A) Embedding layer}
In the beginning, the embeddings for tokens and POS were trained from scratch. Here, I used my own built up vocabulary from the training data, containing also the special tokens for START, END, UNKNOWN and PADDING. With this, I also could vary the embedding size. Later, I switched to pretrained \emph{GloVe} embeddings for the token to help the model to understand the relation between the words easier from the start. I utiized the GloVe embeddings that are based on 6 billion tokens embedded in 300 dimensions. I added the special tokens \emph{<SOS>}, \emph{<END>}, \emph{<PAD>} and \emph{<UNK>} with random intializations to its vocabulary. Since the embeddings are already pretrained, while the rest of my model is randomly initialized, I froze them for the first 10 epochs. After that, they are fine-tuned together with the rest of the model.

\subsection*{(B) Encoding layer}
The next step was to encode the sentence step by step (word by word). The simplest solution would have been, to just use a LSTM layer and encode the embeddings automatically. In the project however, I wanted see, how the model performs if I feed it different kind of information and encode them differently. In especially, I wanted to see if provide the model explicitely syntactical information by giving it the parts of speech (POS).

Therefore, the model is able to receive word embeddings and POS embeddings. Both embeddings are then encoded (1) a bidirectional LSTM encoder and (2) a bidirectional LSTM encoder for the POS. Furthermore, the word embeddings are additionally encoded with (3) a transformer encoder. These layers are then combined for each step in an LSTM by either concatination or addition, optionally weigthed as following:
\begin{enumerate}
    \item $concat(token\_LSTM, POS\_LSTM, transformer)$
    \item $concat(token\_LSTM, POS\_LSTM)$
    \item $token\_LSTM + POS\_LSTM$
    \item $token\_LSTM + 0.5 * POS\_LSTM$
    \item $token\_LSTM$
\end{enumerate}

\subsection*{(C) Clasifier}
The classifier is built on top of each step of the LSTMs. This allows me to predict an NLS token in every position of the sentence. For this, the classifier finally combines multiple linear layers together with non-linear functions. This reduced the dimension from the output dimension of the combination of the encoders to 14 classes (NLS tokens + <NO\_NLS> token). The number of encoder dimensions differs depending on the way, the encoders are combined. If the number was very high (e.g. for the concatenation of both LSTMs and the transformer), I introduced an additional linear layer.

For all of the different layers, I applied dropout with a probability of 0.2 during training. This should prevent the model from overfitting to the training data.

\subsection*{Input for the model}
As explained before, I needed to feed the model the words aligned with POS. The example sentence \emph{uh-huh i think so and uh} should be represented like this with an output by the model:

\begin{table}[h]
    \centering
    \begin{tabular}{c || c c c c c c c}
        input  tokens & <SOS>  & i        & think    & so       & and & <EOS>    \\
        input POS     & <SOS>  & NN       & VBP      & RB       & CC  & <EOS>    \\
        \hline
        output        & uh-huh & <NO-NLS> & <NO-NLS> & <NO-NLS> & uh  & <NO-NLS>
    \end{tabular}
\end{table}

For this, I first tokenized the sentences, using \emph{nltk}s \emph{word\_tokenize} function. Afterwards, I excluded the NLS, added start-of-sequence and end-of-sequence tokens. This is then aligned with the extracted NLS enriched with <NO\_NLS> tokens if there is no NLS at a specific position.

\section*{Results}
Evaluating the models was quite difficult. Since, it is not a standard classification task, but generation of sequences, classic metrics like \emph{accuracy}, \emph{precision} or \emph{recall} may not yield meaningful value. The whole problem could be more seen as a translation task from 'English without NLS' to 'English with NLS'. That's why, I also inlcuded the \emph{Meteor score}, which is usually used in Machine Translation. Still, compared to Machine Translation, the difference between the source and the target sentence is very small, in most of the cases just one token. The Meteor score will therefore always be very high and differences between models may not be significant looking at this metric.

Furthermore, the placement of NLS (as well as the type of NLS) in a sentence is very subjective. NLS don't carry usually as much semantic meaning as normal words. Additionally, there are no grammar rules, when NLS may be used or not. This leads to the fact that every human may use these NLS in a different way and there is now objective 'truth' to there placement (which is rather the case with a normal grammar). Misplaced NLS by the model based on the training and test data may therfore be actually not misplaced based on human judgement. Consider the following two sentences:
\begin{center}
    (1) that's contrary to \textbf{uh} popular belief you know
\end{center}
\begin{center}
    (2) that's contrary \textbf{uh} to popular belief you know
\end{center}
Sentence 1 is picked from the training data. If the model predicted Sentence 2, all of the metrics would punish the model even though for humans, both sentences sound acceptable. In the end, all of the metrics can only give pointers and hints for this problem, which model might perform better than others. A human evaluation would be definiteley necessary to judge the models performance better.

\begin{table}[h]
    \centering
    \begin{tabular}{c || c c | c | c c c c c c c}
            & \multicolumn{2}{c|}{contain NLS} &               &                         &                &                &                &                &                & \\
            & gold                             & predicted     & $\frac{NLS}{<NO\_NLS>}$ & Accuracy       & Precision      & Recall         & Meteor         & NLS score        \\
        \hline
        (1) & 100                              & 41.31         & 6.51                    & 90.32          & 47.57          & 50.03          & 93.55          & 48.37            \\
        (1) & 36.80                            & 4.63          & 2.70                    & \textbf{96.79} & \textbf{79.19} & \textbf{80.64} & 97.16          & \textbf{49.70}   \\
        (2) & 36.80                            & 4.45          & 2.70                    & 96.70          & 78.59          & 79.97          & 97.18          & 49.66            \\
        (3) & 36.80                            & \textbf{4.98} & 2.70                    & 96.67          & 78.52          & 79.88          & \textbf{97.20} & 49.65            \\
        (4) & 36.80                            & 4.63          & 2.70                    & 96.69          & 78.56          & 79.93          & 97.19          & 49.66            \\
        (5) & 36.80                            & 4.86          & 2.70                    & 96.66          & 78.46          & 79.82          & 97.19          & 49.64
    \end{tabular}
    \caption{Evaluation of each model}
\end{table}

I tested the models with two different datasets. In one dataset, I kept all of the utterances, independently if they contained NLS. For this, the test split consists of around 36.8\% utterances that contain at least one NLS. In the second dataset, I filtered out all sentences that don't contain any NLS. The proportion is therfore 100\%.
Since the test data differs from dataset to dataet, the metrics achieved with them can't be compared directly.

The reason for this was that the models generated almost no NLS at all using the unfiltered dataset. The results consisted only of 4.4\% to 5\% of sentences that contain NLS (compared to 36.8\% of the gold data). Using the filtered data set, 41\% of the generations contained NLS (compared to 100\% of the gold data).

For the unfiltered dataset, all of the models produce very high metrics. The accuracy, as well as the Meteor score don't differ significantly across the models. Also the margin for the precision and recall lie under 0.6\% and 0.8\% respectiveley, which doesn't seem very significant.

My model has very high scores for all the metrics \emph{Accuracy}, \emph{Precision} and \emph{Recall}, even though the predicted sentences are not really good. The reason for that is only a very small proportion of the slots are NLS, while the biggest proportion is the <NO\_NLS> token. Therefore, my model that often only predicts <NO\_NLS> tokens for all of the slots performs very well. To make this problem better visible, I introduced a new metric, called the \emph{NLS score}. It is a weighted accuracy that weights <NO\_NLS> tokens antiproportinally to its occurence in the test corpus. More specifically the weights are calculated as following for the whole test corpus:

\begin{align}
    W_{NLS}     & = \frac{number\ of\ NO\_NLS\ tokens}{number\ of\ slots} \\
    W_{NO\_NLS} & = 1 - W_{NLS}
\end{align}

The weights are then applied to each token $t$ of a sentence, depending the gold label. If the gold label was a <NO\_NLS> token, $W_{NO\_NLS}$ is applied, otherwise $W_{NLS}$. The sum of the weighted scores is then normalized over the length of the sentence $n$ and averaged over all sentences.
\begin{align}
    A_{W,sentence} & = \frac{W_{NLS} * \sum_{}^{t}{correct\ NLS} + W_{NO\_NLS} *\sum_{}^{t}{correct\ NO\_NLS}}{t} \\
    A_W            & = \frac{\sum_{}^{n}A_{WS}}{n}
\end{align}

Finally, the NLS score is the mean of the resulting weighted accuracy with the average 'normal' accuracy.

\begin{equation}
    NLS\ score = \frac{A_W + A}{2}
\end{equation}

This new metric takes the inbalance of <NO\_NLS> and NLS tokens better into account. It is also flexible in respect to a changing ratio, since it could also handle the opposite case if the biggest portion of slots were NLS tokens. Furthermore, it let's me compare the filtered and unfiltered dataset a little better.

As expected the NLS score is much lower for all different models. But also here, the differences between the models are not significant. All models consistently produce too few sentences with NLS. Even using the filtered dataset yields a similar low NLS score.

Next, I looked at the prediced sentences that contained NLS. Most of the NLS, that were predicted were 'uh' or 'um'. It was hard to see any pattern, in which position they appeared. They appeared mostly in the middle of the sentence or towards the end. However, many of them were predicted before a noun or after conjunctions like 'that' or 'but'. Still, there appeared also a lot of NLS, which I couldn't classify together and understand, why they may have been predicted.

\section*{Discussion}
There are multiple
\end{document}
